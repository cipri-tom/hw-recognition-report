%!TEX root = ../main.tex
\chapter*{Notations}
\addcontentsline{toc}{chapter}{Notations}

This section provides a concise reference for different notation styles and abbreviations that we use thorough this document.
% Apart from the following mathematical notations, we also use special rules for:

\begin{table}[!h]
% \flushleft
\begin{tabular}[t]{c l}
	\multicolumn{2}{l}{\bf Numbers} \\\hline
	\({0,1}\) & The set containing 0 and 1 \\
	\(a\) & A scalar\\
	% \(i = 1..N\) & Variable \(i\) takes values from 1 to N\\
	\(\ve{v}\) & A vector (lowercase)\\
	\(\ve{I}\) & A matrix (uppercase)\\
	\(\mathcal{F}\) & A feature representation \\
	\multicolumn{2}{l}{} \\
	\multicolumn{2}{l}{\bf Indexing} \\\hline
	\(\ve{v}_i\) & Element \(i\) of vector \(\ve{v}\)\\
	\(\ve{I}_{r,c}\) & Element at \(r,c\) in matrix\\
	\(\ve{I}_{r,:}\) & All elements in row \(r\)\\
	\(\ve{I}_{:,c}\) & All elements in column \(c\)\\
	\multicolumn{2}{l}{} \\
\end{tabular}\hspace{1cm}
\begin{tabular}[t]{c l}
	\multicolumn{2}{l}{\bf Other} \\\hline
	\CTPN{} & Name of an architecture\\
	\multicolumn{2}{l}{} \\
	\ds{Dataset} & Name of a dataset\\
	\ds{Model}& or of a model that was \\
	 &trained on that dataset \\
\end{tabular}
\end{table}
